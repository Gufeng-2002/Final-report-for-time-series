\documentclass[12pt]{article}
% \usepackage{setspace}         % For line spacing if needed

\newcommand{\notebox}[1]{
\vspace{.5cm}
\hrule
\vspace{5pt}
{\Large \noindent \textit{Note :}\par}
   #1
\vspace{5pt}
\hrule
\vspace{.5cm} }

\newcommand{\figbox}[4]{
\begin{tcolorbox}[colback=cyan!5!white, colframe=white, boxrule=0mm, sharp corners] % 背景色等已经设置过
    \begin{center}
    \Large \textit{#1}
    \end{center}
    \vspace{.1cm}
   \begin{minipage}[t]{0.35\textwidth}  % 左侧放置图片的 minipage,并垂直居中
    \includegraphics[width=\linewidth, valign=t]{#2} % 用\includegraphics命令来放入图片文件
    \par  
    \vspace{20pt}
    \textit{#3} % 在图片下方放入注解文字
  \end{minipage}
  \hspace{0.03\textwidth}  % 水平间距
  \begin{minipage}[t]{0.62\textwidth}  % 右侧放置文本的 minipage
    {#4} %放入右侧文本
  \end{minipage}
\end{tcolorbox}
}
\input{settings/packages.tex}
% set the gap between the content and margins of page
\geometry{
top = 2.0cm,
bottom = 2cm
}




\title{Temperature forecast for Delhei with SARIMA-improved linear model}
\author{Feng Gu(T00751197), Yishu Liu(T00728937), Haoran He(T00749480)}
\date{\today}

\begin{document}
\maketitle

\begin{center}
    \textbf{\large Abstract}
\end{center}

The present study investigates regional temperature trends in Delhi.
 By applying time series analysis techniques, historical temperature data were analyzed, and various models, 
 including the SARIMA model, standard linear model, and other benchmark forecasting models, 
 were constructed and evaluated for their performance on both training and testing datasets. 
 The results indicate that the SARIMA model with dummy variables outperforms other models in predicting 
 future temperatures. 

Due to page limitations for formal content, most tables and figures referenced in this report 
are included in the appendix. Additionally, the complete workflow, a notable aspect of this study, 
has been made available in the accompanying GitHub repository for transparency and reproducibility.

\textbf{GitHut Repo:} \href{https://github.com/Gufeng-2002/Final-report-for-time-series.git}
{https://github.com/Gufeng-2002/Final-report-for-time-series.git}



\section{Introduction}
\sloppy
The subject of global warming and climate change is gradually becoming one of the significant challenges 
that the world must face. More frequent and intense extreme weather events, such as heat waves, dust storms, 
and floods, have been observed globally \cite{dabhade2021}. 

The issue of climate change has become particularly crucial in large, densely populated cities. 
One such example is Delhi, India. The effects of climate change have intensified in recent years, 
posing challenges to human health, agricultural production, and the environment 
\cite{hussain2024}. 
Therefore, studying the temperature trends in Delhi holds high scientific value and practical significance. 
\sloppy
This study employs time series analysis, enabling the examination of historical temperature data,
comparison of different time series models, and the prediction of future temperature trends using the 
best-fit model. This research aims to provide a comprehensive understanding of temperature trends in the Delhi region,
and to explore and practice a more efficient way to organize and finish the paper writing work.

\section{Data}
\subsection{Source of data}
From Kaggle\footnote{a machine learning community for learners},  
we downloaded our weather data, which is 
the climate data(of shape (1576,5)) about Delhei of India. 
Each record in the dataset contains 5 variables: date, mean temperature, 
humidity, wind speed and mean pressure. The mean temperature is the target varialbe and 
the other variables, except date, are predictors.

\subsection{Preparing and processing the data}
To the raw data, we process it by following the procedure below,
some additional explanation and corresponding results can be found in appendix:
\begin{itemize}
    \item Checking the missing values, if there are, replacing or removing the missing records.
    \item Exploring the distribution for the 4 variables, simple box-plot and hist-plot.
    \item Replacing the abnormal outliers\footnote{outliers were detected using customized algorithm, which could be 
    found in the code of Module} 
    with corresponding moving average value.
    \item Creating dummy variables from the "date" variable: four seasons.   
\end{itemize}

\subsection{ProcessRawData module of Python}
It is notable that these steps are finished in a workflow with module \textit{"ProcessRawData.py"}
\cite{financialriskforecasting},
which has been pushed to the public Git repository. It can be easy
\footnote{only needing to point or change the directory path correctly} 
to repeat all these steps or make 
further adjustments to make it suitable for other work. 

\section{Method}
\subsection{Specifing the desired model}
Before we set a specific model for forecasting \textit{meantemp}, 
we decomposed the \textit{meantemp} using TSL method\cite{fpp3stl}. Becasue
we have daily climate data, we set the season period as 365, assuming 
the same day in each year should have the most similar pattern in Temperature
\footnote{But it is not rigirous, because every 
four-year there is one more day premium and the number of day
is not an accurate "365" of interge.}.

After observing the possible seasonality and trend, we create a assume the model as following:
\[
y_t = \beta_{5 \times 1} X + \beta_{3 \times 1} H + \eta_t
\]
in which:
\[
X = \begin{bmatrix}
    1 & 1 & X_{11} & X_{21} & X_{31}\\
    1 & 2 & X_{12} & X_{22} & X_{32}\\
    & & ... & ... &\\
    1 & t & X_{1t} & X_{2t} & X_{3t} \\
    \end{bmatrix}
\quad H_i = 
\begin{bmatrix}
    1 & 0 & 0\\
    0 & 1 & 0\\
     & ... & \\
    0 & 0 & 0 \\
    \end{bmatrix} =
\begin{cases}
    1, \text{if the season is i} \\
    0, other wise
\end{cases}
\]
There are totally three $H_i$ here to aviod multilinearity 
caused by including intercept. To the $\eta_t$, we assume it follows
a SARIMA or ARIMA model, specificly:
\[
\Phi^P(B^s) \phi^p(B) (1-B^s)^D (1-B)^d \eta_t = 
\Theta^Q(B^s) \theta^q(B) \epsilon_t
\]
where, we set the $s$ equal to 365(days). The searching for peroper order  
of SARIMA((P,D,Q) and (p,d,q)) and the specific claculating are finished
by R language.

\subsection{Comparisions with other models}
In order to assess our model properly, we build five other models 
in table 3 at the same time and use them fit and predict together.

\subsection{ModelFitting module of R}
To fit these models quickly and easily, we choose R to build these models and 
do relevant tests on them and visualize the results. Same to the Python module,
there is a \textit{"ModlFitting.R"} in the repo. To make the workload less, there are
some functions that quickly convert the "table-similar" R object into Latex code of table,
that's why there are many tables in the appendix, similar function is defined for 
saving plot objects quickly.

\section{Result}
The specific settings about models can be found in the R module, here we turn to
our attention to the standard linear models and SARIMA-improved models, especially
changes of significance level for the three dummy variables.

From table 4 and 5 in appendix, we found that models with dummy variabes are always
better in performance than models without that. 
The four varialbes: \textit{time, season\_Autumn, season\_Spring, season\_Summer}
pass the 0.05 significance level test under the 
$H_{0}\footnote{$H_0$: the coefficient is value of 0, namely no influcence from this variable}$
assumption, however, they do not pass the corresponding tests in SARIMA models.

To the reason why these variables become not important in SARIMA models,
one explanation might be that the influences from these four variables can be captured
well by the SARIMA part in the model, and the long-term trend with \textit{time} is also
not imporant to mean temperature, based on the given sample.

Additionally, there is one counterintuitive coefficent: the coefficient
for \textit{season\_Summer} is smaller than that of \textit{season\_Spring}, which should 
not be correct by checking the summary about the average feature value in table 1(appendix).
It might indicate that some predictors in our model take the effect from \textit{season\_Summer},
if we remove these interfering factors, the relationship might be shown correctly, or this is 
the truth of the real word.

Although some coefficients did not pass the significance level test, we can still
use the model to forecast, becasue we are focusing on the relationship between these
variables but the future values of target.

According the table 4 and 7, we compared the performance of these models on
training data and testing data, finally chose the SARIMA with dummy variables
as the model to represent the forecasts, information about this model is shown 
in table 1.
\begin{table}[!h]
    \centering
    \captionsetup{font=small} % Set caption to left-align and smaller font
    \caption{\textit{Summay about the SARIMA model.
    Including the coefficents, tests about residuals from training data,
    and criteria about performance from testing data.}}
    \label{tab:model_summary_combined}
    \begin{tabular}{lccccc}
    \toprule
    \textbf{Metric} & \textbf{ME} & \textbf{RMSE} & \textbf{MAE} & \textbf{MPE} \\
    \midrule
    \multirow{2}{*}{sarima\_dummy} 
        & 0.5447 & 3.0829 & 2.6184 & -0.079 \\
        \cmidrule{2-5}
        & \textbf{MAPE} & \textbf{ACF1} & \textbf{log\_lik} & \textbf{AIC} \\
    \cmidrule{2-5}
        & 12.4737 & 0.8543 & -2369.349 & 4762.699 \\
    \midrule
    \textbf{Coefficient} & \textbf{Estimate} & \textbf{Std. Error} & \textbf{Statistic} & \textbf{P-value} \\
    \cmidrule{1-5}
    ar1            & 0.9898  & 0.0041 & 242.1087 & 0.0000 \\
    ma1            & -0.0953 & 0.0298 & -3.2015  & 0.0014 \\
    ma2            & -0.1798 & 0.0300 & -5.9982  & 0.0000 \\
    humidity       & -0.1363 & 0.0042 & -32.4098 & 0.0000 \\
    wind\_speed    & -0.0291 & 0.0072 & -4.0637  & 0.0001 \\
    meanpressure   & -0.0322 & 0.0076 & -4.2461  & 0.0000 \\
    time           & 0.0021  & 0.0045 & 0.4730   & 0.6363 \\
    season\_Autumn & 0.2608  & 0.5227 & 0.4990   & 0.6179 \\
    season\_Spring & 0.5930  & 0.5235 & 1.1326   & 0.2576 \\
    season\_Summer & 0.4116  & 0.6098 & 0.6751   & 0.4997 \\
    intercept      & 63.9278 & 8.5701 & 7.4594   & 0.0000 \\
    \midrule
    \textbf{Other Metrics} & \textbf{sigma2} & \textbf{log\_lik} & \textbf{AICc} & \textbf{BIC} \\
    \cmidrule{1-5}
    \multirow{2}{*}{sarima\_dummy} & 1.5048 & -2369.349 & 4762.914 & 4826.149  \\
    \cmidrule{2-5}
     & \textbf{lb\_stat} & \textbf{lb\_pvalue} & \textbf{bp\_stat} & \textbf{bp\_pvalue} \\
     \cmidrule{2-5}
     & 1.5524 & 0.2128 & 1.5492 & 0.2133 \\
    \bottomrule
    \end{tabular}
\end{table}

We also visualized the forecasts for all models to make the comparisons more
clear and direct.
\begin{figure}[!h]
    \centering
    \includegraphics[width=.8\textwidth]{images/forecasts_CI90.png}
    \captionsetup{font=small} % Set caption to left-align and smaller font
    \caption{\textit{Forecasts from eight models. 
    Becasue of the assumptions and settings to models,
    we should compare the closes forecasts from SARIMA with forecasts
    from other models.}}
    \label{fig:figure1}
\end{figure}

\section{Discussion}
\subsection{Explanation about the model results}
According the regression results, we found that \textit{humidity, wind speed and mean pressure} have
negative effect on mean temperature, with their increases, the temperature decreases.
In comparison with the winter, the other seasons have higher mean temperature, 
even thought the coefficent of \textit{Spring} and \textit{Summer} might look counterintuitive, 
which could be the task for further exploration.

The autoregression and moving average parts show there are strong autocorrelation
in the mean temperature variable, which could be explained by standard linear model
that considers \textit{time and seasons} variables in some extent.

\subsection{Other useful work and further improvement.}
We have to admit it is not a very rigirous report due to the lack of time and 
the limits of our skills and professional knowledge in coding and Time Series field.

However, this report is a try in using Vscode, Rstudio, Latex entention as a complete
workflow, in which we manage to finish all the work in one system and make the
whole process automatic as much as possible. The complete frame work could be found in the GitHub
repository, including the document frame of Latex.

To make the whole workflow better, i think we can make improvement with the following
aspects:
\begin{itemize}
    \item Learn Time Series forecasting models more
    \item Be familiar with R and Python for Data Science
    \item Be familiar with using Vscode, Rstudio and GitHub for collaboration.
\end{itemize}

\clearpage
% Bibliography section
\begin{thebibliography}{99} % The number specifies the width of the label

    \bibitem{dabhade2021} 
    A. Dabhade, S. Roy, M. S. Moustafa, S. A. Mohamed, R. El Gendy, and S. Barma, 
    ``Extreme Weather Event (Cyclone) Detection in India Using Advanced Deep Learning Techniques,'' 
    \textit{2021 9th International Conference on Orange Technology (ICOT)}, 
    Tainan, Taiwan, 2021, pp. 1--4, 
    doi: \href{https://doi.org/10.1109/ICOT54518.2021.9680663}{10.1109/ICOT54518.2021.9680663}.

    \bibitem{hussain2024} 
    Hussain S., Hussain E., Saxena P., Sharma A., Thathola P., Sonwani S., 
    ``Navigating the impact of climate change in India: a perspective on climate action (SDG13) and sustainable cities and communities (SDG11),'' 
    \textit{Frontiers in Sustainable Cities}, 2024 Jan 23. 
    Available from: 
    \href{https://research-ebsco-com.ezproxy.tru.ca/linkprocessor/plink?id=6c6991da-78bd-3062-a586-5a5d83ba7467}{https://research-ebsco-com.ezproxy.tru.ca}.

    \bibitem{financialriskforecasting}
    A. McNeil. 
    ``Financial Risk Forecasting: R Best Practice,'' 
    \textit{Financial Risk Forecasting Notebook}. 
    Available at: \url{https://www.financialriskforecasting.com/notebook/R/BestPractice.html}. 
    Accessed: November 30, 2024.

    \bibitem{fpp3stl}
    H. Hyndman and G. Athanasopoulos. 
    ``STL Decomposition,'' 
    \textit{Forecasting: Principles and Practice (3rd ed.)}. 
    Available at: \url{https://otexts.com/fpp3/stl.html}. 
    Accessed: November 30, 2024.

\end{thebibliography}





\clearpage
\section{Appendix}

\subsection{Figures}

\begin{figure}[!h]
    \centering
    \includegraphics[width=.8\textwidth]{images/raw_train_data.png}
    \caption{\small \textit{Distribution of the raw data without replacing outliers}}
    \label{fig:figure1}
\end{figure}

\begin{figure}[!h]
    \centering
    \includegraphics[width=.8\textwidth]{images/processed_train_data_box_and_hist_plots.png}
    \caption{\small \textit{Distribution of the processed data after replacing outliers}}
    \label{fig:figure1}
\end{figure}

\begin{figure}[!h]
    \centering
    \includegraphics[width=.8\textwidth]{images/decomposition_plot.png}
    \caption{\small \textit{STL Decomposition of the mean temperature variable 
    (pointing period = 365 days)}}
    \label{fig:figure1}
\end{figure}

\begin{figure}[!h]
    \centering
    \includegraphics[width=.8\textwidth]{images/fitted_values_all_models.png}
    \caption{\small \textit{Fitted values and the true values on training data}}
    \label{fig:figure1}
\end{figure}

\begin{figure}[!h]
    \centering
    \includegraphics[width=.8\textwidth]{images/forecasts_CI90.png}
    \caption{\small \textit{Forecasts with 90\% confidence interval of all models}}
    \label{fig:figure1}
\end{figure}

\begin{figure}[!h]
    \centering
    \includegraphics[width=.8\textwidth]{images/best_model_resid_diagnostic.png}
    \caption{\small \textit{Residual diagnostic plot for SARIMA with dummy variables}}
    \label{fig:figure1}
\end{figure}

\begin{figure}[!h]
    \centering
    \includegraphics[width=.8\textwidth]{images/linear_dummy_model_resid_diagnostic.png}
    \caption{\small \textit{Residual diagnostic plot for standard linear model with dummy variables}}
    \label{fig:figure1}
\end{figure}







\input{sections/appendix_table.tex}


\end{document}